\documentclass[]{elsarticle} %review=doublespace preprint=single 5p=2 column
%%% Begin My package additions %%%%%%%%%%%%%%%%%%%
\usepackage[hyphens]{url}

  \journal{Patterns} % Sets Journal name


\usepackage{lineno} % add
  \linenumbers % turns line numbering on
\providecommand{\tightlist}{%
  \setlength{\itemsep}{0pt}\setlength{\parskip}{0pt}}

\usepackage{graphicx}
\usepackage{booktabs} % book-quality tables
%%%%%%%%%%%%%%%% end my additions to header

\usepackage[T1]{fontenc}
\usepackage{lmodern}
\usepackage{amssymb,amsmath}
\usepackage{ifxetex,ifluatex}
\usepackage{fixltx2e} % provides \textsubscript
% use upquote if available, for straight quotes in verbatim environments
\IfFileExists{upquote.sty}{\usepackage{upquote}}{}
\ifnum 0\ifxetex 1\fi\ifluatex 1\fi=0 % if pdftex
  \usepackage[utf8]{inputenc}
\else % if luatex or xelatex
  \usepackage{fontspec}
  \ifxetex
    \usepackage{xltxtra,xunicode}
  \fi
  \defaultfontfeatures{Mapping=tex-text,Scale=MatchLowercase}
  \newcommand{\euro}{€}
\fi
% use microtype if available
\IfFileExists{microtype.sty}{\usepackage{microtype}}{}
\bibliographystyle{elsarticle-harv}
\ifxetex
  \usepackage[setpagesize=false, % page size defined by xetex
              unicode=false, % unicode breaks when used with xetex
              xetex]{hyperref}
\else
  \usepackage[unicode=true]{hyperref}
\fi
\hypersetup{breaklinks=true,
            bookmarks=true,
            pdfauthor={},
            pdftitle={lmtp: An R Package for Estimating the Causal Effects of Feasible Interventions Based on Modified Treatment Policies},
            colorlinks=false,
            urlcolor=blue,
            linkcolor=magenta,
            pdfborder={0 0 0}}
\urlstyle{same}  % don't use monospace font for urls

\setcounter{secnumdepth}{0}
% Pandoc toggle for numbering sections (defaults to be off)
\setcounter{secnumdepth}{0}


% Pandoc header



\begin{document}
\begin{frontmatter}

  \title{lmtp: An R Package for Estimating the Causal Effects of
Feasible Interventions Based on Modified Treatment Policies}
    \author[a]{Nicholas Williams\corref{1}}
   \ead{niw4001@med.cornell.edu} 
    \author[a]{Ivan Diaz}
  
      \address[a]{Division of Biostatistics, Department of Population
Health Sciences, Weill Cornell Medical College, 402 E. 67th Street, New
York, NY, 10065}
      \cortext[1]{Corresponding Author}
  
  \begin{abstract}
  In many situations, it is impossible to estimate the causal effects of
  a policy intervention using randomization, resulting in a reliance on
  causal inference methods using observational data. The majority of
  these methods consider counterfactual outcomes where exposure is set
  deterministically. However, these treatment effects may not be
  particularly relevant and are often infeasible to bring about,
  especially in the case of continuous exposures. Modified treatment
  policies offer a non-parametric alternative to deterministic treatment
  effects that allow for the study of feasible interventions and offer a
  safegaurd against positivity violations. In this article, we introduce
  the lmtp \texttt{R} package. The lmtp package provides doubly-robust,
  non-parametric estimators of dynamic point-treatment and longitudinal
  modified treatment policies for binary, categorical, and continuous
  exposures with missing outcomes while leveraging ensemble machine
  learning for estimation.
  \end{abstract}
  
 \end{frontmatter}

\hypertarget{introduction}{%
\section{Introduction}\label{introduction}}

Most modern causal inference methods consider the effects of a treatment
on population mean outcomes under interventions that set the treatment
value deterministically. For example, the average treatment effect
considers the hypothetical difference in population mean outcomes if a
binary treatment was applied to all observations versus if it was
applied to no observations. In the case of a categorical or continuous
exposure, it is unlikely any policy intervention could bring this about.
In addition, the estimation of causal effects requires the so called
positivity assumption which states that all observations have a greater
than zero chance of experiencing the exposure levels. This assumption is
often violated when evaluating the effects of deterministic
interventions and is usually exacerbated with longitudinal data.

In response to the aforementioned issues, alternative causal effects
have been defined in the literature that can be formulated to avoid
violations of the positivity assumption{[}INSERT CITATIONS HERE FOR
THESE PAPERS{]}. Of immediate relevance to this article are
\emph{modified treatment policies}. Building off work by {[}INSERT DIAZ
AND VAN DER LAAN{]} Modified treatment policies were first introduced by
{[}INSERT ROTNISKY{]} and were extended to the longitudinal setting by
{[}INSERT DIAZ, WILLIAMS, HOFFMAN{]}. Modified treatment policies are
defined as interventions that can depend on the \emph{natural} value of
the exposure. {[}INSERT SECTION COMMENTING ON STATIC VS DYNAMIC
INTERVENTIONS{]}.

The \texttt{R} package \textbf{lmtp} (available for download from GitHub
at https://github.com/nt-williams/lmtp) implements four estimators for
estimating the effects of modified treatment policies in both the
point-treatment and longitudinal studies. Two of these estimators, a
targeted minimum-loss based estimator (TMLE) and a sequentially
doubly-robust estimator (SDR), are doubly-robust. {[}INSERT EXPLANATION
OF WHAT THIS MEANS{]}. The software allows for static or dynamic
interventions, binary, categorical, and continuous exposures, and binary
and continuous outcomes with right censoring. In addition, \textbf{lmtp}
is capable of estimating deterministic causal effects such as the
average treatment effect. Estimation procedures are carried out using
the Super Learner algorithm by way of the \texttt{sl3} package {[}INSERT
CITATION FOR SL3{]} and cross-fitting is used to maintain \(n^{1/2}\)
convergence {[}INSERT CITATIONS FOR WHY WE USE CROSS-FITTING{]}. In this
article we describe how the \textbf{lmtp} package can be used for
estimating static and dynamic deterministic and modified treatment
policy effects for a variety of research questions. Code examples are
provided.

\hypertarget{results}{%
\section{Results}\label{results}}

\hypertarget{required-data-structure}{%
\subsection{Required Data Structure}\label{required-data-structure}}

\hypertarget{overview-of-function-parameters}{%
\subsection{Overview of function
parameters}\label{overview-of-function-parameters}}

\hypertarget{specifying-policies-of-interest}{%
\subsection{Specifying policies of
interest}\label{specifying-policies-of-interest}}

\hypertarget{using-the-super-learner}{%
\subsection{Using the Super Learner}\label{using-the-super-learner}}

\hypertarget{outcomes-of-interest}{%
\subsection{Outcomes of interest}\label{outcomes-of-interest}}

\hypertarget{example-1-static-binary-trt.-continuous-outcome}{%
\paragraph{Example 1: Static, Binary Trt., Continuous
Outcome}\label{example-1-static-binary-trt.-continuous-outcome}}

\hypertarget{example-2-mtp-longitudinal-continuous-trt.-binary-outcome}{%
\paragraph{Example 2: MTP, Longitudinal, Continuous Trt., Binary
Outcome}\label{example-2-mtp-longitudinal-continuous-trt.-binary-outcome}}

\hypertarget{example-3-dyanmic-deterministic-longitudinal-categorical-trt.}{%
\paragraph{Example 3: Dyanmic Deterministic, Longitudinal, Categorical
Trt.}\label{example-3-dyanmic-deterministic-longitudinal-categorical-trt.}}

\hypertarget{example-4-mtp-survival-outcome-wcensoring-time-varying-confounders}{%
\paragraph{Example 4: MTP, Survival Outcome w/Censoring, Time-varying
confounders}\label{example-4-mtp-survival-outcome-wcensoring-time-varying-confounders}}

\hypertarget{discussion}{%
\section{Discussion}\label{discussion}}

\hypertarget{references}{%
\section*{References}\label{references}}
\addcontentsline{toc}{section}{References}


\end{document}

